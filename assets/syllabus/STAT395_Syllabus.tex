\documentclass[12pt]{article}

\usepackage{fontspec}
\usepackage{geometry}
\usepackage{lastpage}
\usepackage{fancyhdr}
\usepackage{hyperref}

\geometry{top=1in, bottom=1in, left=1in, right=1in, marginparsep=4pt, marginparwidth=1in}

\renewcommand{\headrulewidth}{0pt}
\pagestyle{fancyplain}
\fancyhf{}
\cfoot{\thepage\ of \pageref{LastPage}}

\setlength{\parindent}{0pt}
\setlength{\parskip}{0pt}

% \setromanfont [Ligatures={Common}, Numbers={OldStyle}, Variant=01,
%  BoldFont={LinLibertine_RB.otf},
%  ItalicFont={LinLibertine_RI.otf},
%  BoldItalicFont={LinLibertine_RBI.otf}
%  ]{LinLibertine_R.otf}

\usepackage{tikz}
\def\checkmark{\tikz\fill[scale=0.4](0,.35) -- (.25,0) -- (1,.7) -- (.25,.15) -- cycle;}

\usepackage{xunicode}
\defaultfontfeatures{Mapping=tex-text}

\setromanfont{YaleNew}

\begin{document}

\begin{center}
{\bf MATH/CS 395: Statistical Learning, Fall 2017} \\
Tuesday, Thursday 13:30-14:45 \quad PURH G13\\
\end{center}

\bigskip

\noindent
\begin{tabular}{ l l }
{\bf Instructor:} &  {\bf Taylor Arnold} \\
E-mail: & \href{mailto:tarnold2@richmond.edu}{tarnold2@richmond.edu} \\
Office: & Jepson Hall, Rm 218 \\
Office hours: & Tuesday, Thursday 10:30-12:00 or by appointment
\end{tabular}

\vspace{0.5cm}

\textbf{Computing:} \vspace{6pt}

The focus of this course will be on applied statistics and data analysis
over symbolic mathematics. To facilitate this, nearly every class
assignment and exam will involve some form of computing.
No prior programming experience is assumed or required. \\

We will use the \textbf{R} programming environment throughout the
semester. It is freely available for all major operating systems and
is pre-installed on many campus computers. You can download it and
all supporting files for your own machine via these links:
\begin{center}
\url{https://cran.r-project.org/} \\
\url{https://www.rstudio.com/}
\end{center}
I strongly recommend using your own machine for this course and
bring a laptop or tablet to each class meeting.
The lab computers in Jepson are available and contain some, though
not all, of the required software.

\vspace{0.4cm}

\textbf{Course Website:} \vspace{6pt}

All of the materials and assignments for the course will be posted
on the class website:
\begin{quote}
\url{https://statsmaths.github.io/stat395}
\end{quote}
At the end of the semester, this version of the course
will be archived and available for your reference.

\vspace{0.4cm}

\textbf{GitHub:} \vspace{6pt}

All of your work for this semester will be submitted through GitHub,
the same platform that hosts our website. You'll need to set up a free
account, which we will cover during the week of class.

\vspace{0.4cm}

\textbf{Labs:} \vspace{6pt}

Every course (other than this first one) through Thanksgiving will
have an associated file named lab00.Rmd, with the appropriate class
number replaced for the 00. By noon before the start of the next
class, you must complete the questions contained within the lab
notebook. Assignments will be submitted through GitHub; this
process will explained in more detail during class. \\

During most class meetings, we will do some combination of presenting
your results in small groups or to the class in general. Note that
your presence and attention in class will be an important aspect
of your lab grade.

\newpage

\textbf{Final Project:} \vspace{6pt}

There will also be a final project for this course consisting of
both written and oral components. Details for this project will be
given following Fall Break. You will have a wide flexibility in
selecting an appropriate project.

\vspace{0.4cm}

\textbf{Grades:} \vspace{6pt}

You will receive four letter grades in this course. Three of
these will be aggregate grades covering approximately 1/3 of the
labs each. The fourth will cover the written and oral components
of your final project.

I want to make the grading extremely transparent. Your final
grade will be computed by converting letter grades into numbers
as follows (pluses increase the number by 0.33 and minuses
decrease the number by 0.33):

\begin{center}
\begin{tabular}{c || c}
Numeric Score & Final Grade \\
\hline \hline
4 & A  \\
3 & B  \\
2 & C  \\
1 & D  \\
0 & F
\end{tabular}
\end{center}
These numeric grades will be averaged according to the following
weights:

\begin{itemize}\setlength\itemsep{0em}
\item Final Project, 34\%
\item Labs, 66\% (22\% each)
\end{itemize}

And reading off of the following chart (grades are rounded to the
second digit):

\begin{center}
\begin{tabular}{c || c}
Numeric Score & Final Grade \\
\hline \hline
3.84 - 4.00 & A  \\
3.50 - 3.83 & A- \\
3.17 - 3.49 & B+ \\
2.84 - 3.16 & B  \\
2.50 - 2.83 & B- \\
2.17 - 2.49 & C+ \\
1.84 - 2.16 & C  \\
1.50 - 1.83 & C- \\
0.00 - 1.49 & F
\end{tabular}
\end{center}

\vspace{0.4cm}

\textbf{Exams:} \vspace{6pt}

This course has no exams, final or otherwise.

\newpage

\textbf{Weekly Topics:} \vspace{6pt}

The semester is broken up into roughly three parts, with the
first week focused on setting up software and reviewing the
required prerequisites. Weeks 2-6 offer a a basic introduction
to machine learning covering principals such as complexity,
over fitting, and regularization. In the second part, we spend
4 weeks covering topics in Natural Language Processing. In
the final 4 weeks we cover topics from computer vision.

\vspace{0.5cm}

\def\labelitemi{}
\def\labelitemii{}

\begin{itemize}\setlength\itemsep{0em}
\item WEEK 01 - Introduction to R, RMarkdown, and Graphics \\
\item WEEK 02 - ML I: Linear Regression and Model Metrics
\item WEEK 03 - ML II: Incorporating Non-Linear Effects
\item WEEK 04 - ML III: Model Regularization
\item WEEK 05 - ML IV: Adaptive, Local Models
\item WEEK 06 - ML V: Using Dense Neural Networks \\
\item WEEK 07 - NLP I: Lexical Frequencies
\item WEEK 08 - NLP II: Sentiment Analysis
\item WEEK 09 - NLP III: Learning and Using Dependency Structures
\item WEEK 10 - NLP IV: Word Embeddings and Recurrent Neural Networks (RNN's) \\
\item WEEK 11 - CV I: Histograms, Filters, and Other Manual Features
\item WEEK 12 - CV II: Convolutional Neural Networks (CNN's)
\item WEEK 13 - CV III: Transfer Learning
\item WEEK 14 - CV IV: Style Transfer and Generative Models (GAN's)
\end{itemize}

\end{document}





